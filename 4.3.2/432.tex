\documentclass[a4paper,12pt]{article}

%% Language and font encodings
\usepackage[english, russian]{babel}
\usepackage[utf8x]{inputenc}
\usepackage{blindtext}
\usepackage[T1]{fontenc}
\usepackage[T2A]{fontenc}
\usepackage[a4paper,top=1.5 cm,bottom=2cm,left=3cm,right=3cm,marginparwidth=1.75cm]{geometry}
%% Useful packages
\usepackage{amsmath, amssymb}
\usepackage{wrapfig}
\usepackage{graphicx}
\usepackage[usenames]{color}
\usepackage[T1]{fontenc}
\usepackage{tikz}
\usetikzlibrary{arrows}
\usetikzlibrary{decorations.pathreplacing}
\usepackage[T2A]{fontenc}
\usepackage{color}
\usepackage{circuitikz} 
\graphicspath{{pic/}}
\definecolor{water} {rgb} {0.667, 0.855, 1}
\usepackage{pgfplots}
\usepackage{pgfplotstable}
\usetikzlibrary{circuits}
\usetikzlibrary{circuits.ee}
\usetikzlibrary{circuits.ee.IEC}
\usetikzlibrary{circuits.logic.IEC}
\usetikzlibrary{intersections}

\title{ДИФРАКЦИЯ СВЕТА НА ЗВУКОВОЙ ВОЛНЕ В ЖИДКОСТИ}
\date{Работа 4.3.2}
\author{Мамедов Маил, Б01-006}
\begin{document}
	\begin{center}
		\LARGE{Работа 4.3.2}\\[0.2cm]
		\LARGE{Дифракция света на ультразвуковой волне в жидкости}\\[0.2cm]
		\large{Мамедов Маил Б01-006}\\[0.2cm]
	\end{center}  
	
	
	\section{Аннотация}
	
	В работе изучается дифракция света на синусоидальной акустической решетке и наблюдается фазовая решетка методом темного поля.
	
	С помощью оптической скамьи, осветителя, двух длиннофокусных объективов, кюветы с жидкостью, кварцевого излучателя с микрометрическим винтом, генератора звуковой частоты, линзы, горизонтальной нити на рейтере и микроскопа.
	
	
	\section{Теоретические сведения}
	
	В работе используются оптическая скамья, осветитель, два длиннофокусных объектива, кювета с жидкостью, кварцевый излучатель с микрометрическим винтом, генератор звуковой частоты, линза, горизонтальная нить на рейтере, микроскоп. 
	
	При прохождении ультразвуковой волны через жидкость в ней возникают периодические неоднородности коэффициента преломления, создается фазовая решетка, которую мы считаем неподвижной ввиду малости скорости звука относительно скорости света. Показатель
	преломления n изменяется по закону:
	
	\begin{equation}\label{}
		n = n_0 (1 + m \cos \Omega x)
	\end{equation}
	
	Здесь $ \Omega = 2 \pi / \Lambda $ --- волновое число для ультразвуковой волны, $ m $ --- глубина модуляции $ n $ $ (m \ll 1 $).
	
	Положим фазу $ \phi $ колебаний световой волны на передней стенке кюветы равной нулю, тогда на задней поверхности она равна:
	
	\begin{equation}\label{}
		\phi  = k n L = \phi_0 (1 + m \cos \Omega x)
	\end{equation}
	
	Здесь $ L $ --- толщина жидкости в кювете, $ k = 2 \pi / \lambda $ --- волновое число для света.
	
	После прохождения через кювету световое поле есть совокупность плоских волн, распространяющихся под углами $ \theta $, соответствующими максимумам в дифракции Фраунгофера:
	
	\begin{minipage}{0.47\textwidth}
		\begin{equation}\label{}	
			\Lambda \sin \theta_m = m \lambda
		\end{equation}
		
		Этот эффект проиллюстрирован на рисунке 1.
		
		Зная положение дифракционных максимумов, по формуле (1) легко определить длину ультразвуковой волны, учитывая малость $ \theta $: $ \sin \theta \approx \theta \approx l_m /F  $, где $ l_m $ --- расстояние от нулевого до последнего видимого максимума, $ F $ --- фокусное расстояние линзы. Тогда получим:
		
		\begin{equation}\label{}
			\Lambda = m \lambda F/ l_m 
		\end{equation}
		Скорость ультразвуковых волн в жидкости, где $ \nu $ --- частота колебаний излучателя:
		
		\begin{equation}\label{}
			v = \Lambda \nu 
		\end{equation}
		
		
	\end{minipage}	
	\begin{minipage}{0.47\textwidth}
		\begin{center}
			
			\includegraphics[width=0.9\textwidth]{1.png}
			
			{рис. 1. Дифракция световых волн на акустической решетке}
		\label{diff}
		
	\end{center}
	
\end{minipage}	

\section{Результаты измерений и обработка данных}

\subsection{Определение скорости ультразвука по дифракционной картине}

Схема установки приведена на рисунке 2. Источник света Л через светофильтр Ф и конденсор К освещает горизонтальную щель $ S $, находящуюся в фокусе объектива $ O_1 $. После объектива параллельный световой пучок проходит через кювету С перпендикулярно акустической решетке, и дифракционная картина собирается в фокальной плоскости объектива $ O_2 $ , наблюдается при помощи микроскопа М.

Предварительную настройку установки произведем в соответствии с инструкцией с зеленым фильтром, далее в работе используется красный.

\begin{center}
\includegraphics[width=0.7\textwidth]{2.png}

рис. 2. Схема для наблюдения дифракции на акустической решетке
\label{shema1}
\end{center}

Параметры установки: фокусное расстояние объектива $  O_2  $ $ F = 30 $ см, одно деление винта микроскопа составляет 4 мкм, погрешность измерений примем равной  $ \sigma = $ 2 деления, или 8 мкм.

Исследуем изменения дифракционной картины на зеленом свете. При увеличении частоты УЗ-генератора и приближении к 1,17 МГц проявляется дифракционная решетка: расстояние между максимумами растет.

Измерим положения $ x_m $ дифракционных максимумов с помощью микроскопического винта для четырех частот. Результаты измерений занесены в таблицы 1-4 ниже. На основе каждой таблицы построены графики зависимости $ x_m (m) $, они изображены на рисунках 3-6. Коэффициенты углов наклонов прямых для всех зависимостей сведены в таблицу 5. 

\

%\newline

\begin{minipage}{0.47\textwidth}
\begin{center}
	\includegraphics[width=0.7\textwidth]{pic1.jpg}
	
	Полученная картина
\end{center}
\end{minipage}	
\begin{minipage}{0.47\textwidth}
\begin{center}
	\begin{center}
		\begin{tikzpicture}[scale = 1.0]
			\begin{axis}[
				axis lines = left,
				ylabel = {$x_m, \ мкм$},
				xlabel = {$m$},
				minor grid style={black, line width=0.05pt},
				major grid style={solid,black, line width=0.3pt},
				xmin=-5, xmax=5,
				ymin=0, ymax=3500,
				ymajorgrids = true,
				xmajorgrids = true,
				yminorgrids = true,
				xminorgrids = true,
				minor tick num = 4
				]
				\addplot+[only marks ] plot[error bars/.cd, y dir=both, y explicit]
				coordinates {
					(-4,450)
					(-3,730)
					(-2,1080)
					(-1,1370)
					(0,1730)
					(1,2060)
					(2,2360)
					(3,2700)
					(4,2950)
					
				};
				
				\addplot[blue, domain=-8:8]{330*x+1700};
			\end{axis}
			
		\end{tikzpicture}
		График зависимости $x_m(m)$ при частоте генератора $\nu$ = 1,17 МГц	
	\end{center}
	
\end{center}
\end{minipage}	

\

%\newline

\begin{center}
\begin{tabular}{|c|c|c|c|c|c|c|c|c|c|}
	\hline
	$m$ &-4&-3&-2&-1&0&1&2&3&4\\
	\hline
	$x_m, \ \text{мкм}$ &450&730& 1080&1370&1730&2060&2360&2700&2950\\
	\hline
\end{tabular}

\

%\newline

Таблица 1. Измерение координаты m-ого максимума $x_m$ дифракционной картины при частоте генератора $\nu$ = 1,17 МГц
\end{center}

\

%\newline

\begin{minipage}{0.47\textwidth}
\begin{center}
	\includegraphics[width=0.7\textwidth]{pic1.jpg}
	
	Полученная картина
\end{center}
\end{minipage}	
\begin{minipage}{0.47\textwidth}
\begin{center}
	\begin{center}
		\begin{tikzpicture}[scale = 1.0]
			\begin{axis}[
				axis lines = left,
				ylabel = {$x_m, \ мкм$},
				xlabel = {$m$},
				minor grid style={black, line width=0.05pt},
				major grid style={solid,black, line width=0.3pt},
				xmin=-4, xmax=4,
				ymin=0, ymax=3500,
				ymajorgrids = true,
				xmajorgrids = true,
				yminorgrids = true,
				xminorgrids = true,
				minor tick num = 4
				]
				\addplot+[only marks ] plot[error bars/.cd, y dir=both, y explicit]
				coordinates {
					
					(-3,50)
					(-2,460)
					(-1,1150)
					(0,1590)
					(1,2110)
					(2,2710)
					(3,3280)
					
					
				};
				
				\addplot[blue, domain=-8:8]{530*x+1650};
			\end{axis}
			
		\end{tikzpicture}
		График зависимости $x_m(m)$ при частоте генератора $\nu$ = 1,82 МГц	
	\end{center}
	
\end{center}
\end{minipage}	

\

%\newline

\begin{center}
\begin{tabular}{|c|c|c|c|c|c|c|c|}
	\hline
	$m$ &-3&-2&-1&0&1&2&3\\
	\hline
	$x_m, \ \text{мкм}$ &50&460& 1150&1590&2110&2710&3280\\
	\hline
\end{tabular}

\

%\newline

Таблица 2. Измерение координаты m-ого максимума $x_m$ дифракционной картины при частоте генератора $\nu$ = 1,82 МГц
\end{center}	

\newpage

\

%\newline

\begin{minipage}{0.47\textwidth}
\begin{center}
	\includegraphics[width=0.7\textwidth]{pic2.jpg}
	
	Полученная картина
\end{center}
\end{minipage}	
\begin{minipage}{0.47\textwidth}
\begin{center}
	\begin{center}
		\begin{tikzpicture}[scale = 1.0]
			\begin{axis}[
				axis lines = left,
				ylabel = {$x_m, \ мкм$},
				xlabel = {$m$},
				minor grid style={black, line width=0.05pt},
				major grid style={solid,black, line width=0.3pt},
				xmin=-4, xmax=4,
				ymin=0, ymax=3500,
				ymajorgrids = true,
				xmajorgrids = true,
				yminorgrids = true,
				xminorgrids = true,
				minor tick num = 4
				]
				\addplot+[only marks ] plot[error bars/.cd, y dir=both, y explicit]
				coordinates {
					
					(-3,460)
					(-2,790)
					(-1,1230)
					(0,1610)
					(1,2030)
					(2,2540)
					(3,3000)
					
					
				};
				
				\addplot[blue, domain=-8:8]{450*x+1650};
			\end{axis}
			
		\end{tikzpicture}
		График зависимости $x_m(m)$ при частоте генератора $\nu$ = 1,55 МГц	
	\end{center}
	
\end{center}
\end{minipage}	

\

%\newline

\begin{center}
\begin{tabular}{|c|c|c|c|c|c|c|c|}
	\hline
	$m$ &-3&-2&-1&0&1&2&3\\
	\hline
	$x_m, \ \text{мкм}$ &460&790& 1230&1610&2030&2540&3000\\
	\hline
\end{tabular}

\

%\newline

Таблица 3. Измерение координаты m-ого максимума $x_m$ дифракционной картины при частоте генератора $\nu$ = 1,55 МГц
\end{center}

\

%\newline

\begin{minipage}{0.47\textwidth}
\begin{center}
	\includegraphics[width=0.7\textwidth]{pic2.jpg}
	
	Полученная картина
\end{center}
\end{minipage}	
\begin{minipage}{0.47\textwidth}
\begin{center}
	\begin{center}
		\begin{tikzpicture}[scale = 1.0]
			\begin{axis}[
				axis lines = left,
				ylabel = {$x_m, \ мкм$},
				xlabel = {$m$},
				minor grid style={black, line width=0.05pt},
				major grid style={solid,black, line width=0.3pt},
				xmin=-3, xmax=3,
				ymin=0, ymax=5500,
				ymajorgrids = true,
				xmajorgrids = true,
				yminorgrids = true,
				xminorgrids = true,
				minor tick num = 3
				]
				\addplot+[only marks ] plot[error bars/.cd, y dir=both, y explicit]
				coordinates {
					
					
					(-2,440)
					(-1,1500)
					(0,2790)
					(1,3730)
					(2,5070)
					
					
					
				};
				
				\addplot[blue, domain=-8:8]{1100*x+2600};
			\end{axis}
			
		\end{tikzpicture}
		График зависимости $x_m(m)$ при частоте генератора $\nu$ = 3.96 МГц	
	\end{center}
	
\end{center}
\end{minipage}	

\

%\newline

\begin{center}
\begin{tabular}{|c|c|c|c|c|c|}
	\hline
	$m$ &-2&-1&0&1&2\\
	\hline
	$x_m, \ \text{мкм}$ &440&1500& 2790&3730&5070\\
	\hline
\end{tabular}

\

%\newline

Таблица 4. Измерение координаты m-ого максимума $x_m$ дифракционной картины при частоте генератора $\nu$ = 3.96 МГц
\end{center}


\begin{minipage}{0.47\textwidth}
Из зависимости наклона графиков от частоты рассчитаем скорость звука в воде по формулам (2.4) и (2.5). Откуда получаем, что скорость звука равна $$1513 \pm 35\ \text{м/с},$$ что соответствует табличным данным в пределах погрешности измерений и эксперимента -- 1490 м/с. 
\end{minipage}
\begin{minipage}{0.47\textwidth}
\begin{center}
	\begin{tikzpicture}[scale = 1.0]
		\begin{axis}[
			axis lines = left,
			ylabel = {$b,\ мкм$},
			xlabel = {$\nu, \ МГц$},
			minor grid style={black, line width=0.05pt},
			major grid style={solid,black, line width=0.3pt},
			xmin=1, xmax=4.25,
			ymin=200, ymax=1200,
			ymajorgrids = true,
			xmajorgrids = true,
			yminorgrids = true,
			xminorgrids = true,
			minor tick num = 3
			]
			\addplot+[only marks ] plot[error bars/.cd, y dir=both, y explicit]
			coordinates {
				
				
				(1.17,330)
				(1.55,450)
				(1.82,520)
				(3.96,1100)
				
				
				
			};
			
			\addplot[blue, domain=-8:8]{300*x-40};
		\end{axis}
		
	\end{tikzpicture}
	График зависимости $b_m(\nu)$
\end{center}

\end{minipage}
\newpage
\section{Определение скорости ультразвука методом темного поля}

Для наблюдения акустической решетки используется метод темного поля, который заключается в устранении центрального дифракционного максимума с помощью непрозрачного экрана. Схема установки показана на рисунке.

\begin{center}

\includegraphics[width=0.7\textwidth]{3.png}
	
Схема для наблюдения дифракции методом темного поля
\label{shema2}
\end{center}

Приставим к задней стенке (для светового луча) кюветы стеклянную пластинку с миллиметровыми делениями; сфокусируем микроскоп на изображение пластинки. Определим цену деления окулярной шкалы микроскопа, совместив ее с миллиметровыми делениями: в 1 делении миллиметровой шкалы убирается 1 большое деление окулярной. Значит, цена деления окулярной шкалы: $ C = $ 1 мм.

\begin{center}
	
	\includegraphics[width=0.7\textwidth]{pic3.jpg}
	
	%Схема для наблюдения дифракции методом темного поля
	%\label{shema2}
\end{center}

Без применения метода темного поля звуковая решетка не наблюдается. Закроем нулевой максимум горизонтальной нитью. Таким образом, осевая составляющая фазово-модулированной волны поглощается, а боковые остаются без изменения. Получившееся поле: 

\begin{equation}\label{}
f(x) = \dfrac{im}{2} e^{i\Omega x} +  \dfrac{im}{2} e^{-i\Omega x} = im \cos \Omega x  I(x) = m^2 \cos ^2 \Omega x = m^2 \dfrac{1 + \cos ^2 2 \Omega x}{2}
\end{equation}

Отсюда получаем, что расстояние между темными полосами есть $ \Lambda/2 $.

%\newpage

Проведем измерение длины ультразвуковой волны, приняв ошибку равной цене деления окулярной шкалы. В таблице 6 содержатся количество маленьких делений окулярной шкалы N (цена деления $ C = 1 $), соответствующее $ n $ темным полосам акустической решетки.
Формулы для расчета длины волны ультразвука $ \Lambda $ и скорости распространения $ v $ в воде:
\begin{equation}\label{}
\Lambda/2  = NC/(n - 1),  \qquad v = \nu\Lambda
\end{equation}

Картина наблюдения получилась только при частоте $3,04 \text{ МГц}$, с помощью этих данных можно определить скорость звука и длину волны. 

 $\Lambda = 1.29\pm0.04 \ \text{мм}$ и $v = 1570 \pm 80\ \text{м/с}$. 

\

%\newline


\begin{center}
	\includegraphics[width=0.7\textwidth]{pic4.jpg}
	
	Наблюдаемая картина при частоте\\ 3,04 МГц
\end{center}



\section{Вывод}

В работе изучена дифракция света на акустической решетки, рассчитаны длина волны ультразвука и скорость его распространения в воде. Решетка наблюдалась методом
темного поля.

Ошибка при определении $ \Lambda $ и $ v $ не превышает 3\%. Согласно справочным данным, при комнатной температуре скорость ультразвуковой волны в воде составляет примерно 1490 м/с. Значения, полученные экспериментально, с достаточной точностью соотносятся с ними.

Ошибка при таком определении скорости звука больше, чем в первой части работы, и
составляет около 5\%. Сами значения тоже получились больше.
Кроме того, мы смогли получить звуковую решетку только на одной частоте из всего диапазона, наиболее вероятно, что это связано с тем что установка не центрирована.

\end{document}